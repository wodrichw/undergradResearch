\documentclass[11pt,leqno]{article}
\usepackage{latexsym,amsfonts,amsmath,amssymb,verbatim,multicol,endnotes,amsthm}

\usepackage[utf8]{inputenc}
%\newcommand{\Z}{\mathbb{Z}}
%\newcommand{\R}{\mathbb{R}}
%\newcommand{\N}{\mathbb{N}}
%\newcommand{\Q}{\mathbb{Q}}

\newtheorem{maintheorem}{Theorem}
\renewcommand{\themaintheorem}{\Alph{maintheorem}}
\newtheorem{mainconjecture}[maintheorem]{Conjecture}
\newtheorem{Theorem}{Theorem}[section]
\newtheorem{Conjecture}[Theorem]{Conjecture}
\newtheorem{Corollary}[Theorem]{Corollary}
\newtheorem{Lemma}[Theorem]{Lemma}
\newtheorem{Proposition}[Theorem]{Proposition}
\newtheorem{Definition}[Theorem]{Definition}
%
%\newtheorem{Discuss}[Theorem]{Discussion Problem}%[section]
%\newtheorem{Practice}[Theorem]{Practice Problem}
\newtheorem{Example}[Theorem]{Example}

\setlength{\textwidth}{13.8cm}
\setlength{\textwidth}{6in}
\setlength{\topmargin}{0.5cm}
\setlength{\headheight}{0cm}
\setlength{\headsep}{0cm}
\setlength{\topskip}{0cm}
\setlength{\textheight}{21.7cm}
\setlength{\oddsidemargin}{1cm}
\setlength{\evensidemargin}{1cm}
\setlength{\labelwidth}{0cm}
\setlength{\leftmargin}{0cm}
\setlength{\listparindent}{0cm}
\setlength{\baselineskip} {15pt}

%\voffset = -10pt \topmargin = 0pt \headheight = -20pt \headsep = 0pt
%
%\textheight = 750pt
%\textwidth = 485pt
%\linespread{1.2} \hoffset = -20pt \marginparwidth = 0pt
%\marginparpush = 0pt \evensidemargin = 0pt \oddsidemargin = 0pt
%


\newcommand{\sss}{\hspace{0.1in}}
\newcommand{\op}{\subseteq^{\scriptstyle op}}
\newcommand{\cl}{\subseteq^{\scriptstyle cl}}
\newcommand{\tfr}[1]{\mbox{\rm rk}_{\Z}(#1)}
\newcommand{\Ann}[1]{\mbox{\rm Ann}\,_{\Z #1}}\,
\newcommand{\Res}[2]{\mbox{\rm Res}\,^{\Z #1}_{\Z #2}}\,
\newcommand{\Cent}[1]{\mbox{\rm Cent}\,_{\Z #1}}\,
\newcommand{\Sig}[2]{\Sigma({#1},{#2})}
\newcommand{\Sigc}[2]{\Sigma({#1},{#2})^c}
\newcommand{\ov}[1]{\overline{#1}}
\def\qed{\hfill $\bullet$}
\def\proof{\noindent {\em Proof:}\ }
\def\solution{\noindent {\em Solution:}\ }
\def\set-up{\noindent {\em Set-Up:}\ }
\def\conclusion{\noindent {\em Conclusion:}\ }
\def\answer{\noindent {\em Answer:}\ }

\def\'{^\prime}
\def\"{^{\prime\prime}}
\def\<{\langle}
\def\>{\rangle}
\def\ra{\rightarrow}

\def\C{\mathbb C}
\def\R{\mathbb R}
\def\Q{\mathbb Q}
\def\E{\mathbb E}
\def\H{\mathbb H}
\def\S{\mathbb S}
\def\F{\mathbb F}
\def\Z{\mathbb Z}
\def\im{{\rm im \,}}
\def\ra{\rightarrow}
\def\lra{\longrightarrow}
\def\la{\leftarrow}
\def\<{\langle}
\def\>{\rangle}
\def\lcm{\mbox{lcm}}
%\def\'{\,^\prime}
\def\"{\,^{\prime\prime}}
\def\Ra{\Rightarrow}
\def\iff{\Leftrightarrow}
\def\ddx{\frac{d}{dx}\ }
\def\ddt{\frac{d}{dt}\ }
\def\dydx{\frac{dy}{dx}}
\def\dudx{\frac{du}{dx}}
\def\ddyddx{\frac{d^2y}{dx^2}}
\def\intab{\int_a^b}
\newcommand{\dd}[2]{\frac{d{#1}}{d{#2}}}
\def\dis{\displaystyle}
\def\sinOverx{\sin\left(\frac{1}{x}\right)}
\def\arcsec{\mathrm{arcsec}\,}
\def\arccsc{\mathrm{arccsc}\,}
\def\arccot{\mathrm{arccot}\,}
\def\lHop{l'H\^{o}pital's Rule\ }
\def\LHop{L'H\^{o}pital's Rule\ }
\def\ker{\mathrm{ker}\,}
\newcommand{\G}{\Gamma}
\newcommand{\pres}[2]{\langle {#1}\ |\ {#2} \rangle}
\newcommand{\gpres}[1]{\langle {#1} \rangle}


\usepackage[utf8]{inputenc}
\usepackage[english]{babel}
 
\usepackage{amsthm}
 
\theoremstyle{definition}
\newtheorem{definition}{Definition}[section]
 
\theoremstyle{remark}
\newtheorem*{remark}{Remark}

\usepackage{graphicx}


\begin{document}
\title{Split Lemma}
\author{The Assemblers}
\date{\today}
\maketitle

\subsection{Defintions}
\begin{definition}[Parity]
The number of ones in a binary string.
\end{definition}

\begin{definition}[Word]
\end{definition}

\begin{definition}[Word Tree]
\end{definition}

\begin{definition}[Word Group]
\end{definition}

\begin{definition}[Word List]
\end{definition}

\begin{definition}[Cardinality of a Word Group]
\end{definition}

\begin{definition}[Exclusion]
\end{definition}

\begin{definition}[Word Multiplication]
\end{definition}

\begin{definition}[Doubling Operation]
	$\dot{2}g = \{wa | w \in g, a \in \{0, 1\}\}$
\end{definition}

\begin{definition}[Child]
	$\bar{C}(A) = \{ab | a \in A, b \in \{0,1\}\}$
\end{definition}

\begin{definition}[Parent]
	$P(A) = \{a | ab \in A, b \in \{0,1\}$
\end{definition}

\begin{definition}[Key Word]
	Given a word group A with associated exclusion set $E$, $k$ is a key word of A if $\forall e \in E, e\notin kb, b \in \{0, 1\}$.
\end{definition} 

\begin{definition}[Exclusion Operator]
	The Exclusion function $\Upsilon$ takes an unbalanced word list $\tilde{A}$ and a list of exclusions $E$, and gives a new list with all the exlusions removed.
	$\Upsilon(\tilde{A}, E) = \{a | a \in \tilde{A} \sss and \sss \forall e \in E, e \dot{\notin} \tilde{A} \}$
\end{definition}

\begin{definition} [Subword]
$\dot{\in}$
\end{definition}


\section{Introduction}
Define ${\vec{o} \atop I}_{k-1}$ to be a word group that is of the form,
$${\vec{o} \atop I}_{k-1} = \begin{bmatrix}
	0 & 0 & \dots & & 0 \\
	1 & 0 &\dots && 0 \\ 
	0 & 1 & \dots && 0 \\
	\vdots && \ddots &&\vdots \\
	0 & 0 & \dots && 1
	\end{bmatrix}.$$
That is ${\vec{o} \atop I}_{k-1}$ takes the form of an Identity matrix with a zero vector hat. ${\vec{o} \atop I}$ has word length $k-1$ and list length $k$, that is $| {\vec{o} \atop I}_{k-1} | = k$.
We aim to prove by induction that all word groups of the form ${\vec{o} \atop I}_{k-1}$ will have $k-2$ exlcusions and they will be of the form $\{11, 101, 1001, \dots, 10^{k-3}1\}$. Then we will show that this implies that every word list of this form will always generate two children lists, i.e yield a split. Given that our word tree will always have a branch of the form $\vec{o} \atop I$ for every $n$, our word tree will generate a split in every level. 

\section {Base Case}
Consider the simplest word group, 
$$ {\vec{o} \atop I}_{0} = \begin{bmatrix}
	0 \\ 
	1
\end{bmatrix} $$
With list length of $n = 2$, note it has $n - 2$ exclusions, (i.e. no exclusions).
Also consider the next simplest word group, 
$$ {\vec{o} \atop I}_{1} =  \begin{bmatrix}
	0 & 0 \\
	1 & 0 \\
	0 & 1 
\end {bmatrix} $$
With list length of $n = 3$, note it has $n-2$ exclusions, namely the exclusions $\{1, 1\}$.
	
Thus the base case, $ {\vec{o} \atop I}_{2} $ assumptions are met.

\section {Induction Step}
Assume that $ E_{k-2} = \{11, 101, 1001, \dots, 10^{k-3}1\} $ are $ k-2 $ exclusions of $ {\vec{o} \atop I}_{k-1} $.
The goal is to show that ${\vec{o} \atop I}_{k}$ has $k-1$ exclusions, and that they are of the form, 
$$ E_{k-1} = \{11, 101, 1001, \dots, 10^{k-3}1, 10^{k-2}1\}. $$

Consider the new list $\dot{2} ({\vec{o} \atop I}_{k-1})$, with length $2k$.
By defintion of how $\dot{2} ({\vec{o} \atop I}_{k-1})$ is generated, 
	$$ E_{k-2} = \{11, 101, 1001, \dots, 10^{k-3}1 \} \sss \dot{\in} \sss \dot{2}( {\vec{o} \atop I}_{k-1}).$$

Then, consider the operation,	
	$$ \Upsilon( \dot{2} ({\vec{o} \atop I}_{k-1}), E_{k-2}) = {\vec{o} \atop I}_{k} \cap \{10^{k-2}1\}. $$

By defintion ${\vec{o} \atop I}_{k} \cap \{10^{k-2}1\}$ is unbalanced since it has parity 2, $|10^{k-2}1| - |\vec{0}| = 2$. Thus $10^{k-2}1$ defines an exclusion for ${\vec{o} \atop I}_{k}$.
Therefore ${\vec{o} \atop I}_{k}$ has $k-1$ exclusions of the form $E_{k-1} = \{11, 101, 1001, \dots, 10^{k-3}1, 10^{k-2}1\}$. 

$\therefore$	A split will occur in every level n, $\forall n \in \mathbb{N}.$


\end{document}